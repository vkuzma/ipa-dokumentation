%!TEX root = ../dokumentation.tex

\begin{table}
\section{Reflexion vom 10.5.2012}
\begin{tabular}{| l | p{12cm} |}
    \hline
    Tätigkeiten &
    \begin{itemize}
        \item Testumgebung vorbereiten
    \end{itemize}  \\
    \hline
    Probleme &
    Um automatisierte Tests zu erstellen musste ich das Script ein wenig modifizieren. Die Änderungen haben den Code verschönert
    \hline
    Erreichte Ziele &
    \hline 
    Reflexion &
    Als ich mit dem Erstellen der Testumgebung begonne habe versuchte ich als erstes die original Umgebung des Scripts nachzubauen. Das heisst, ich habe die Dateistruktur nachgebildet. Anschliessend habe ich die Dateistruktur in der Testumgebung ein bisschen vereinfacht. Das Testresultat sollte davon nicht betroffen sein. Der Grund der Vereinfachung war auch, dass wenn jemand einen neuen Test dazu schreiben will, sich besser zurechtfinden kann. Eine unübersichtliche Testumgebung verleiht nur zu Fehlern, die nichts mit dem zu testendem Script zu tun haben. 
    \hline
\end{tabular}
\end{table}

\clearpage
