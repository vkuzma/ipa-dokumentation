%!TEX root = ../dokumentation.tex

\begin{table}
\section{Reflexion vom 30.4.2012}
\begin{tabular}{| l | p{12cm} |}
    \hline
    Tätigkeiten &
    \begin{itemize}
        \item Initialisierung der Dokumentation 
        \item Erfassen des Auftrags
        \item Analyse des Ist-Zustands
        \item Analyse der Aufgabenstellung
    \end{itemize}  \\
    \hline
    Probleme &
    Es haben sich keine Probleme ergeben. \\
    \hline
    Erreichte Ziele &
    Es wurde alle Ziele ausser der ''Analyse der Aufgabenstellung'' erreicht. Es bin mit der Analyse noch nicht ganz zufrieden. \\
    \hline
    Reflexion &
    Heute habe ich mit meiner IPA begonnen. Ich hatte so gut wie keine Schwirigkeiten zu beginnen. 
    Als erstes habe ich die Dokumentation initialisiert. Das heisst ich habe mir eine Vorlage in Latex zusammengestellt, damit ich mich auf den Inhalt konzentrieren kann und nicht auf das Layout.
    Als nächstes habe ich eine Grobplannung für mein Projekt erstellt. Die detailierte Planung werde ich erst erstellen können, wenn ich die Analyse abgeschlossen habe.
    Mit der Analyse des Ist-Zustands habe ich auch schon begonnen. Bis jetzt bin ich noch auf keine Probleme gestossen. Es sind jediglich ein paar Optimierungsideen bei den ''pre-commit hooks'' aufgetaucht. Diese werde ich noch mit meinem Vorgesetzten besprechen müssen.  
    Um 16.00 bis 16.30 hatte ich mein erstes Gespräch mit meinem Experten. Ich konnte sicher auf seine Fragen antworten, was bestätigt, dass ich die Aufgabenstellung richtig verstanden habe. \\
    \hline
\end{tabular}
\end{table}
\clearpage

