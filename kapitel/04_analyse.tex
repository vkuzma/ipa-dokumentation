%!TEX root = ../dokumentation.tex
\section{Analyse des Ist-Zustands}
Momentan werden unsere Djangoprojekte beim commiten (versionieren mit hilfen von git) mit sognennten ''pre-commit hooks'' analysiert. 
Dabei wird der Quellcode auf diverse Aspekte überprüft. Bei nicht bestehender Überprüfung wird eine Meldung ausgegeben.\\
Diese Methode hat bisher die Fehlerquate um einiges verringert. Es ist jedoch immer noch möglich einen Fehler in die Produktion einzuspielen.  
Folgende Fehlerfälle können trotz pre-commit hooks eintreffen. 
\begin{itemize}
    \item Man commited direkt über Github.
    \item Das requirements.txt für pip ist nicht vollständig.
\end{itemize}
Derzeitig wird der Quellcode nach folgenden Zeichenkette überprüft:
\begin{itemize}
    \item ''import pdb'' - *.py: 
        pdb ist der Python Debugger. Aus Sicherheitsgründen darf dieser nicht in der Produktion verwendet werden.
    \item ''import ipdb'' - *.py:
        Gleiche Begründung wie ''import pdb''.
    \item ''print'' - *.py:
        Konsolenausgabe sind in der Produktion unerwünscht.
    \item ''console.log'' - *.js:
        Konsolenausgabe sind in der Produktion unerwünscht.
    \item ''debugger'' - *.js:
        Das Keyword debugger ist mir im Javascript nicht bekannt.
    \item 
\end{itemize}
Derzeitig wird der Quellcode mit folgenden tools überprüft:
\begin{itemize}
    \item jshint - *.js:
        jshint ist ein Code Qualitätstool, dass den Quellcode nach Fehlern und unschönen Programmiertechniken durchsucht.
    \item pyflakes - *.py:
        pyflakes überprüft den Quellcode nach potenziellen Fehlern.
    \item pep8 - *.py:
        pep8 ist ein Tool welches den Quellcode auf das Einhalten des pep8 Standards überprüft.
    \item sass - *.scss:
       sass ist eine Erweiterung von CSS3. Sass wir von uns nicht mehr verwendet. Diese üperprüfung ist somit nicht nötig. 
\end{itemize}

\subsection{Fehlererkennung}
Jede Djangoapplikation sendet im Fehlerfall automatisch ein Email an eine vorgebeben Email-Adresse mit einem ausführlichen Bericht. Darunter zählen jedoch nur Pythonfehler. Diese Funktionalität wird vom django Framework bereitgestellt. Wenn ein Fehler im settings.py auftritt, wird keine Fehlermeldung gesendet. Der Fehler bleibt unentdeckt bis jemand die Htmlseite aufruft. \\
Generell wird ein Fehler erst entdeckt wenn dieser ausgelöst wird. Das heisst, wenn ein Fehler nicht auf der lokalen Testumgebung ausgelöst und behoben werden kann, so wird mit zimmlicher Wahrscheinlichkeit der Besucher der Webapplikation den Fehler auslösen. \\
Fehler, die durch Serverprobleme verursacht werden, werden nicht speziel abgefangen oder kontrolliert, da diese relativ wenig vorkommen. 

\section{Gewünschter Soll-Zustand}
Die Probleme, die mit der jetztigen Qualitätssicherung auftreten sollen so weit wie möglich verhindert werden. Die meist auftretenden Fehler sollen weit weg vom Kunden entdeckt werden. Das heisst die Überprüfung müsste automatisiert und ständig auf einem externen System laufen.
\clearpage
