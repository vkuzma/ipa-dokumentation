%!TEX root = ../dokumentation.tex
\section{Funktionsumfang}
Die Aufgabenstellung ist im Kapitel 1 detailiert beschrieben. \\
Es wurde eine Liste mit Funktionen erstellt, die für das erreichen des Projektziel notwendig sind.
\begin{itemize}
    \item Das Script soll täglich mittels Cronjob ausgelöst werden.
    \item Klonen eines Projekt aus GitHub. 
    \item Erstellen einer virtuellen Umgebung mittels virtualenvwrapper.
    \item Installieren des Projekts und deren Abhängigkeiten.
    \item Aufsetzen einer mysql-Datenbank. 
    \item Synchronisieren der Datenbank mit den Djangomodellen.
    \item Ausführen der Datenbankmigrationen.
    \item Überprüfung des Quellcodes anhand der vorgegebenen Validierungskriterien (siehe Abschnitt Validierungskriterien).
    \item Starten der Djangoapplikation
    \item Alle Berichte werden in eine Logdatei geschrieben. 
    \item Senden eines Berichts als Email an eine vordefinierte Adresse.
    \item Entfernen der Datenbank aus der lokalen Umgebung.
    \item Entfernen des überprüften Projekt und der virtuellen Umgebung aus der lokalen Umgebung.
\end{itemize}


\section{Validierungskriterien}
Folgende Kriterien müssen erfüllt sein, damit ein Djangoprojekt erfolgreich validiert wird:
\begin{itemize}
    \item Codeüberprüfung durch pep8 - *.py
    \item Codeüberprüfung durch pyflakes - *.py
    \item Codeüberprüfung durch jshint - *.
    \item Der Code darf die folgenden Zeichenfolgen nicht enthalten: ''console.log'', ''debugger'' - *.js
    \item Der Code darf die folgenden Zeichenfolgen nicht enthalten: ''print'', ''import pdb'', ''import ipdb'' - *.py
\end{itemize}

\section{Kriterien für die Erweiterbarkeit}
Damit das Script auch bequem erweitert werden kann, muss auf ein paar Dinge besonds geachted werden. \\
Darunter gehören eine modulare Aufteilung der einzelnen Prüfungsprozesse und eine Kapselung des Basiscodes. \\
Mit Einhaltung folgender Kriterien sollte das möglich sein:
\begin{itemize}
    \item Die Prüfungsscripts sind in einem separatem Order gespeichert.
    \item Der Bezug auf die Prüfungsscripts wird nur in einer Datei gehalten. 
    \item Das Layout für den Bericht, der über Email versendet wird, wird in einer separaten Datei gehalten.
    \item Der Prüfungsbericht soll so gestaltet sein, dass die Resultate auch bei sehr vielen Projekten oder sehr vielen Prüfungsprozessen noch überschauber ist.
\end{itemize}

\clearpage
\section{der Prüfungsberichts}
Der Prüfungsbericht wird nach Abschluss der Überprüfung aller Projekte als Email versendet. Im Bericht sollen alle Diagnosen der Projekte aufgezählt sein.
Die Herausforderung dabei ist es, den Bericht so zu gestalten, dass er auch bei mehreren Projekte oder Prüfungsprozessen noch übersichtlich bleibt. 


