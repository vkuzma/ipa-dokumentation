\section{Funktionalität des Scripts}
\subsection{Testcases auswählen}
Man kann für jedes Projekt entweder eine Liste der Testfälle erstellen, die ausgeführt werden sollen oder eine Liste der Testfälle, die nicht ausgeführt werden sollen.
So kann man für jedes Projekt die Testfälle individuell zusammenstellen.

\subsection{Fehler}
Jede Funktion im Script wird auf Fehler überprüft. Das Script sollte unter keinen Umständen zum absturz kommen. 
Im Fehlerfall werden die restlichen Testfälle übersprungen und der Bericht wird noch erstellt und versendet. \\
Jeder Testfall hat 2 Fehler, die geworfen werden können: 
\begin{itemize}
    \item CriticalError - kritischer Fehler, führt dazu dass die restlichen Testfälle einer Djangoapplikation ausgelassen werden, wird im Report rot dargestellt 
    \item NonCriticalError - nicht kritischeer Fehler, der Fehler wird zwar vermekrt, fürt aber nicht zum Abbruch, die restlichen Testfälle werden weiterhin ausgeführt, wird im Report orange dargestellt
\end{itemize}

\section{Bezogene Infomrationsquellen}
Währen der Implementation war ich auf Informationen im Web angewiesen. Alle Informationen, die ich bennötige, konnte ich von der officiellen Pythonreferenz entnehmen. 
Anbei ist eine Auflistung des Themas und der dazugehörigen Quelle, die ich für das Projekt benötigte:


