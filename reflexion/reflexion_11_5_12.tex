%!TEX root = ../dokumentation.tex

\begin{table}
\section{Reflexion vom 11.5.2012}
\begin{tabular}{| l | p{12cm} |}
    \hline
    Tätigkeiten &
    \begin{itemize}
        \item Testumgebung vorbereiten 
        \item Test durchführen 
        \item Anwendung korrigieren
    \end{itemize}  \\
    \hline
    Probleme &
    Die Gestaltung der Testumgebung erwies sich als schwirig. Um einen nachvollziehbaren Test zu erstellen, wollte ich mit Unittests arbeiten, die nicht direkt mit den einzelnen Testfällen fungieren. Das Problem dabei war nur, dass ich so die einzlnen Testfälle nicht testen konnte. Da die einzelnen Testfälle stark von einender varrierten, ist es sehr schwirig einen nachvollziehbaren Test für alle zu erstellen. Damit aber auch gewährleistet wird, dass die realen Testfälle funktionierten, habe ich mich entschieden 3 Kundenprojekte als Versuchsobjekte zu benutzen. Dabei hatte ich das Problem, dass ich bei solch einem Test keine definitive Vorhersage machen kann. Die Projekte sind mir jedoch bekannt und ich kann deshalb abschätzen, welche Testfälle mit hoher Wahrscheinlichkeit korrekt und welche nicht ausgeführt werden. Allein die Unittest garantieren, die funktionsfähigkeit der Scriptbasis. 

    \hline
    Erreichte Ziele &

    \hline 
    Reflexion &
    Ich habe heute den ganzen Tag an den Tests gearbeitet. Dieser Arbetischritt erwies sich für mich als viel schwiriger als die Planung und Implementierung selbst. Das Script, dass ich geschrieben habe ich selbt ein Überprüfungsscript für andere Djangoprojekte. Die Tests die ich für das Scripts vorbereitet habe, sollen zumindest die Grundfunktionalität des Scripts überprüfen. Ich kamm heute ein bisschen in Stress, weil ich meinen Zeitplan einhalten möchte. Ich denke ich werde für die kommenden Tage noch etwas länger an den Tests arbeiten müssen.  
    \hline
\end{tabular}
\end{table}

\clearpage
