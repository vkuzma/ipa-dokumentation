%  --------------------------------------------------------------------------
%  IPA Qualitätssicherung für Djangoprojekte Dokumentation
%  Created by Vladimir Kuzma on 2012-04-30.
%  --------------------------------------------------------------------------

%  --------------------------------------------------------------------------
%  Latex Document Settings
%  --------------------------------------------------------------------------

\documentclass[
11pt, % Schriftgrösse
a4paper, % A4 Papier
BCOR10mm, % Absoluter Wert der Bindekorrektur, z.B. BCOR1cm
DIV14, % Satzspiegel festlegen siehe
       % http://www.ctex.org/documents/packages/nonstd/koma-script.pdf
footsepline = false, % Trennlinie zwischen Textkörper und Fußzeile
                     % bei normalen Seiten
headsepline, % Trennlinie zwischen Kopfzeile und Textkörper
             % bei normalen Seiten
oneside, % Zweiseitig
openright,
halfparskip, % Europäischer Satz mit Abstand zwischen den Absätzen
abstracton, % inkl. Abstract
listof=totocnumbered, % Abb.- und Tab.verzeichnis im Inhaltsverzeichnis
bibliography=totocnumbered % Lit.zeichnis in Inhaltsverzeichnis aufnehmen
]{scrreprt}

\usepackage[automark]{scrpage2} % Gestaltung von kopf- und Fußzeile
\usepackage[ngerman]{babel}
\usepackage[ngerman]{translator}
\usepackage{tocbasic}
\usepackage[utf8]{inputenc}
\usepackage{lmodern} % Latin Modern
\usepackage[T1]{fontenc}
\usepackage{hyphenat}
\usepackage{ae} % Schöne Schriften für PDF-Dateien
\usepackage{lscape}

% Tradmark
\def\TTra{\textsuperscript{\texttrademark}}

%1.5 Zeilenabstand
\usepackage[onehalfspacing]{setspace}

% Festlegung des Seitenstils (scrpage2)
\pagestyle{scrheadings}
\clearscrheadfoot
\automark[chapter]{section}

% \lehead{\sffamily\upshape\headmark}
% \cehead{}
% \rehead{}
% \lefoot[\pagemark]{\upshape \pagemark}
% \cefoot{}
% \refoot{}
% \lohead{}
% \cohead{}
\lohead{\sffamily\upshape\headmark}
\lofoot{}
\cofoot[\today]{\today}
\rofoot[\pagemark]{\scshape \pagemark}

% Surround parts of graphics with box
\usepackage{boxedminipage}

% Package for including code in the document
\usepackage{listings}

% If you want to generate a toc for each chapter (use with book)
\usepackage{minitoc}
\usepackage{longtable}

% Abkürzungsverzeichnis erstellen.
\usepackage[printonlyused]{acronym}

% schöne Tabelle zeichnen
\usepackage{booktabs}
\renewcommand{\arraystretch}{1.4} %Die Zeilenabstände in Tabellen angepasst.

% für variable Breiten
\usepackage{tabularx}

% Durchgestrichener Text
\usepackage[normalem]{ulem} %emphasize weiterhin kursiv

% This is now the recommended way for checking for PDFLaTeX:
\usepackage{ifpdf}

\usepackage{eurosym}

\usepackage{natbib}

\usepackage{paralist}

\usepackage{array,ragged2e}

\usepackage[hyperfootnotes=false]{hyperref}
\hypersetup{
  bookmarks=true,         % show bookmarks bar?
  unicode=true,           % non-Latin characters in Acrobat’s bookmarks
  pdftoolbar=true,        % show Acrobat’s toolbar?
  pdfmenubar=true,        % show Acrobat’s menu?
  pdffitwindow=true,      % window fit to page when opened
  pdfstartview={FitH},    % fits the width of the page to the window
  pdftitle={IPA Dokumentation},   
  pdfauthor={Marius Küng},
  pdfsubject={Dokumentation IPA yatplaner},
  pdfcreator={TeX Live 2009},
  pdfproducer={pdfTeX, Version 3.1415926-1.40.10},
  pdfnewwindow=true,      % links in new window
  colorlinks=true,       % false: boxed links; true: colored links
  linkcolor=black,          % color of internal links
  citecolor=black,        % color of links to bibliography
  filecolor=magenta,      % color of file links
  urlcolor=cyan          % color of external links
  % linkcolor=black,          % color of internal links
  % citecolor=black,        % color of links to bibliography
  % filecolor=black,      % color of file links
  % urlcolor=black          % color of external links
}

\ifpdf
    \usepackage[pdftex]{graphicx}
\else
    \usepackage{graphicx}
\fi

\makeatletter 
\let\orgdescriptionlabel\descriptionlabel 
\renewcommand*{\descriptionlabel}[1]{% 
  \let\orglabel\label 
  \let\label\@gobble 
  \phantomsection 
  \edef\@currentlabel{#1}% 
  %\edef\@currentlabelname{#1}% 
  \let\label\orglabel 
  \orgdescriptionlabel{#1}% 
} 
\makeatother 

%  --------------------------------------------------------------------------
%  Start Document
%  --------------------------------------------------------------------------
\title{Dokumentation IPA Qualitätssicherung für Djangoprojekte}

\author{IPA in Applikationsentwicklung\\
    \\
    Auszubildender - Vladimir Kuzma\\
	Auftraggeber - allink GmbH\\
    Projektleiter - Silvan Spross\\
    Experte - Roman Fischer\\
    Chefexperte -  \\
    Durchführungsort - allink GmbH\\
	\\
	Informatikschule ZLI }

\date{30.04. - 16.05.2012}

\begin{document}
    \ifpdf
        \DeclareGraphicsExtensions{.pdf, .jpg, .tif}
    \else
        \DeclareGraphicsExtensions{.eps, .jpg}
    \fi

    \pagenumbering{Alph}
  
    \maketitle

    \pagenumbering{arabic}
  
    \tableofcontents
    
    \part{Umfeld und Ablauf}
         \chapter{Aufgabenstellung}
             %!TEX root = ../dokumentation.tex
\section{Titel der Facharbeit} 
Kontinuierliche Qualitätssicherung für Web Projekte in Django
    
\section{Thematik}
Es soll eine Software in Python erstellt werden, welche kontinuierlich die Qualität von einer konfigurierbaren Menge Django Projekte überprüft.

\section{Klassierung}
    
\begin{itemize}
    \item Prozessautomatisierung 
    \item UNIX / Linux
    \item Python / Ruby
\end{itemize}
    
\section{Durchführungsblock}
Startblock 10: 16.04.2012 - 20.04.2012\\
IPA-Durchführung: 30.04.2012 - 16.05.2012\\
Einreichung bis: Montag, 16.03.2012\\
    
\section{Ausgangslage}
Bei allink besteht seit Mitte 2010 ein Entwicklungs- und Deployment-Prozess welcher sich in den letzten eineinhalb Jahren etabliert hat. Die Entwickler sparen sich dadurch pro Projekt einige Stunden Arbeit und die Projekte sind zudem auch nach längere Zeit noch wartbar. Dank der Abhängigkeitsdefinitionen pro Projekt, können Entwickler schnell in einem anderem Projekt einsteigen oder in einem alten Projekt Fehlerkorrekturen vornehmen. Seit einem halben Jahr wird zudem die Codequalität mit Hilfe von einer automatischen Überprüfung der Codekonvention auf den Rechnern der Entwickler erhöht.

Trotz all diesen Massnahmen ist es möglich, dass Entwickler ein Projekt unabsichtlich verunreinigen. Im Extremfall führt dies dazu, dass das Projekt bei einem anderen Entwickler oder sogar in der Produktion nicht mehr lauffähig ist.

Mit dieser Arbeit soll ein System zur automatischen Kontrolle der erwähnten Faktoren erstellt werden.

\section{Detaillierte Aufgabenstellung}
Es soll eine Software in Python erstellt werden, welche kontinuierlich die Qualität von einer konfigurierbaren Menge Django Projekte überprüft. Dazu sollen die Projekte jeweils täglich aus dem Versionierungssystem geklont, Projektspezifische Abhängigkeiten installiert und nach gewissen Anforderungen getestet und abschliessend ein projektübergreifender Report per E-Mail versendet werden.

Diese Anforderungen sind:

- Installierbarkeit der projektspezifischen Abhängigkeiten
Erklärung: Jedes Projekt hat projektspezifische Abhängigkeiten, welche sich über ein Packetmanagement installieren lassen. Bei dieser Anforderung soll pro Projekt geprüft werden, ob diese Abhängigkeitsliste komplett und korrekt ist.

- Codekonventionen nach PEP8
Erklärung: Im ''Python Enhancement Proposal 8'' sind die grundlegenden Stil-Richtlinien für Python-Quellcode definiert. Bei dieser Anforderung soll pro Projekt geprüft werden, dass diese Stil-Richtlinien eingehalten wurden.

- Überprüfung der Syntax
Erklärung: Überprüft werden sollen sämtliche Python und Javascript Files z.B. mit pyflakes und jshint.

- Initialisierung der Datenbank

- Durchführen sämtlicher Datenbankmigrations-Files

- UnitTests

Treten bei den Tests Fehler auf, soll dies in einem E-Mail festgehalten und an eine Sammeladresse gesendet werden.

Eine weitere Anforderung an die Software ist, dass die einzelnen Überprüfungen der Anforderungen Modular aufgebaut sind und so in Zukunft weitere Anforderungen hinzugefügt werden können. Zudem kommt hinzu, dass die neu zu schreibende Applikation mit dem Versionsverwaltungssystem git verwaltet werden soll. Dabei sollen zwei Branches geführt werden: ''develop'' und ''master''. Wobei der ''master'' immer den Stand der Produktion widerspiegelt und im ''develop'' weiterentwickelt wird.

Es sollen wo möglich bestehende Tools zur Überprüfung der oben genannten Anforderungen verwendet werden. Dies bedeutet für den Lernenden, dass er überwiegend die Software zum Koordinieren und zusammenführen der einzelnen Tests schreiben muss.

Nicht Bestandteil dieser Arbeit sind:

- VirtualMachine zur Entwicklung
- Bereitstellung des Zielsystems
- Einpflegen aller Projekte

Nebst der begleitenden IPA Dokumentation wird keine zusätzliche Dokumentation gefordert.

\section{Mittel und Methoden}
Folgende Technologien sind zwingend zu verwenden:

\begin{itemize}
    \item  Python 2.x
    \item  Django
\end{itemize}

Der Lernende muss sich zudem während der Entwicklung an die Firmenstandards halten, welche im IT Wiki zugänglich sind.
    
\section{Vorkenntnisse}
Dem Lernenden sind alle genannten Technologien bereits bekannt. Seit Beginn seines Praktikums im Mai 2010 setzt er sich damit auseinander.
    
\section{Vorarbeiten}
Die Testumgebung wird als VirtualMachine Image mit einem unserem Produktivsystem am nächsten kommenden System bespielt und dem Lernenden zur Verfügung gestellt.    

\section{Neue Lerninhalte}
Wie erwähnt sind dem Lernenden alle Technologien bereits bekannt. Das Know-how ist zudem bei allink ausreichend vorhanden. 

\section{Arbeiten in den letzten 6 Monaten}
Der Lernende hat diverse Webseiten mit den oben genannten Technologien umgesetzt.

         \chapter{Tagesjournal}
            %!TEX root = ../dokumentation.tex
\section{Reflexion vom 30.4.2012}

Heute habe ich mit meiner IPA begonnen. Ich hatte so gut wie keine Anfangsschwirigkeiten. \\
Als erstes habe ich die Dokumentation initialisiert. Das heisst ich habe mir eine Vorlage in Latex zusammengestellt, damit ich mich auf den Inhalt konzentrieren kann und nicht auf das Layout.
Als nächstes habe ich eine Grobplannung für mein Projekt erstellt. Die detailierte Planung werde ich erst erstellen können, wenn ich die Analyse abgeschlossen habe.
Mit der Analyse des Ist-Zustands habe ich auch schon begonnen. Bis jetzt bin ich noch auf keine Probleme gestossen. Es sind jediglich ein paar Optimierungsideen bei den "pre-commit hooks" aufgetaucht. Diese werde ich noch mit meinem Vorgesetzten besprechen müssen. \\ 
Um 16.00 bis 16.30 hatte ich mein erstes Gespräch mit meinem Experten. Ich konnte sicher auf seine Fragen antworten, was bestätigt, dass ich die Aufgabenstellung richtig verstanden habe. 

            %!TEX root = ../dokumentation.tex
\section{Reflexion vom 2.5.2012}

\begin{table}
\begin{tabular}{| l | p{10cm} |}
    \hline
    Tätigkeiten &
    \begin{itemize}
    \item Arbeitsplanung vervollständigen
    \item Prüfungskriterien bestimmen
    \item Kriterien für die Erweiterbarkeit definieren
    \item Prozessdiagramm erstellen
    \item Prüfungsbericht gestalten
    \item Beginn mit der Definition der Modulstruktur
\end{itemize}  \\
    \hline
    Probleme & 
    Es haben sich keine bemerkenswerte Probleme ergeben. \\
    \hline
    Erreichte Ziele &
    Laut der Terminplanung wurden alle Ziele am heutigen Tag erreicht. \\
    \hline 
    Reflexion &
    Ich habe heute meine ersten Meilenstein erreicht. Es wahr teilweise etwas mühsam so oft zwischen den verschiedenen Arbeiten zu wechseln. Manchmal sind mir wieder Sachen eingefallen, die einen anderen Arbeitsschritt betreffen. Ich kann mir gut vorstellen, dass ich noch im Verlauf der Arbeit wieder zu der Definition der Prüfungskriterien oder zum Gestalten des Prüfungsbericht zurück muss. Trotzdem bin ich bin mit meinem heutigen Fortschritt zufrieden. \\
    \hline
\end{tabular}
\end{table}

\clearpage

            %!TEX root = ../dokumentation.tex

\begin{table}
\section{Reflexion vom 3.5.2012}
\begin{tabular}{| l | p{12cm} |}
    \hline
    Tätigkeiten &
    \begin{itemize}
        \item Optimierung der Flussdiagramme  
        \item Optimierung des Klassendiagramms
        \item Start der Implementierung
    \end{itemize}  \\
    \hline
    Probleme & 
    Die Flussdiagramme und das Klassendiagramm habe ich optimiert nachdem ich diese mit meinem Fachvorgesetzten angeschaut hatte. Da ich eine variable anzahl von Testfällen und Projkte in meinem Diagrammen berücksichtigen muss, 
    ist mir der erste Entwurf nicht so gut gelungen. \\
    \hline
    Erreichte Ziele &
    Es wurde alle Tagesziele erreicht. \\
    \hline 
    Reflexion &
    Heute habe ich mit der Implementierung des Scripts begonnen. Die Implementieren hat mir bis jetzt am meisten Spass gemacht. Ich bin desswegen weiter gekommen als ich gedacht hätte. Man kann bereits ein Projekt klonen 
    und ein paar leere Tests laufen lassen. Ich bekomme langsam einen Eindruck des fertigen Scripts. \\
    \hline
\end{tabular}
\end{table}

\clearpage

            %!TEX root = ../dokumentation.tex

\begin{table}
\section{Reflexion vom 4.5.2012}
\begin{tabular}{| l | p{12cm} |}
    \hline
    Tätigkeiten &
    \begin{itemize}
        \item Implementierung des Scripts 
    \end{itemize}  \\
    \hline
    Probleme & 
    Ich habe während der Implementierung des Script bereits mit der Umsetzung des Auswertungsmails begonnen. Laut der Planung wäre es eigentlich noch zu früh. Ich denke aber, dass es mehr Sinn macht wenn die Ausgabe tabellaririsch angezeigt wird, so wie im Auswertungsmail gedacht. So kann ich auch besser die Funktionalität des Scripts überprüfen. Der Arbeitsablauf ist sommit effektiver.\\ 
    \hline
    Erreichte Ziele &
    Es wurden keine Ziele erreicht. Es sind für heute auch keine Ziele angestanden. \\
    \hline 
    Reflexion &
    Ich habe weiter an der Implementieren gearbeitet. Ich musste während dem Implementieren manchmal mein Klassendiagramm anpassen. Gewisse Bezeichnungen haben mir nicht mehr gefallen. Beim Programmieren musste ich oft zwischen den einzelnen Teilaufgaben wechseln, weil sich alles ein bisschen gegenseitig beeinflusst. Nur so konnte ich die Funktionalitäten gegenseitig Testen. \\ 
    \hline
\end{tabular}
\end{table}

\clearpage

            %!TEX root = ../dokumentation.tex

\begin{table}
\section{Reflexion vom 4.5.2012}
\begin{tabular}{| l | p{12cm} |}
    \hline
    Tätigkeiten &
    \begin{itemize}
        \item Implementierung des Scripts 
    \end{itemize}  \\
    \hline
    Probleme & 
    Es gab keine Probleme. \\
    \hline
    Erreichte Ziele &
    Es wurden keine Ziele erreicht. Es war für heute auch keine Ziele angestanden. \\
    \hline 
    Reflexion &
    Ich habe weiter an der Implementieren geareitet. Ich bin heute gut vorankommen. Der Report sieht schon mal gut aus. Ich kann mich jetzt mehr auf die Testfälle konzentrieren. Ich denke ich werde den Terminplan bis zum nächsten Meilenstein einhalten können. \\   
    \hline
\end{tabular}
\end{table}

\clearpage

        \chapter{Einleitung}
            %!TEX root = ../dokumentation.tex
\section{Projekorganisation}

\begin{figure}[!ht]
\begin{center}
    \includegraphics[width=0.5\textwidth,angle=0]{./grafiken/organigram.pdf}
\end{center}
\end{figure}
\footnotetext{Eigene Darstellung}

\section{Projektbeschreibung}
Es soll eine Software in Python erstellt werden, welche kontinuierlich die Qualität von einer konfigurierbaren Menge Django Projekte überprüft. Dazu sollen die Projekte jeweils täglich aus dem Versionierungssystem geklont, Projektspezifische Abhängigkeiten installiert und nach gewissen Anforderungen getestet und abschliessend ein projektübergreifender Report per E-Mail versendet werden.

\section{Rahmenbedingung}
Die Anwendung soll mit folgenden Mitteln realisiert werden:

Technologien
\begin{itemize}
    \item Python 

\end{itemize}

Anwendungen
\begin{itemize}
    \item pep8
    \item pyflakes
    \item jshint
\end{itemize}

Systemen
\begin{itemize}
    \item Mac OS X
    \item Debian lenny
\end{itemize}
    
\section{Systemumgebung}
\begin{itemize}
    \item Betriebsystem (Projekumsetzung): Mac OS X
    \item Editor: vim
    \item Dokumentationshilfsmittel: LaTeX
    \item Entwicklungsumgebung: Python
    \item localer Testserver: Debian Server in einer VM (Virtualbox) 
    \item Versionierung: Git, auf Github veröffentlicht
\end{itemize}

\section{Projektplannungsmethode}
Ich habe mich für die Projektplanungsmethode IPERKA entschieden. \\

\subsection{Begründung}
Für ein kleines Projekt reicht IPERKA meiner Meinung ausreichend aus. Die Planungs- und die Realisierungsphase ist klar getrennt. Es sind alle wichtigen Arbeitsschritte vorhanden, die eine saubere Struktur und ausführliche Dokumentation erlauben. 

\clearpage

    \part{Projekt}
         \chapter{Analyse}
            %!TEX root = ../dokumentation.tex
\section{Analyse des Ist-Zustands}
Momentan werden unsere Djangoprojekte beim commiten (versionieren mit hilfen von git) mit sognennten ''pre-commit hooks'' analysiert. 
Dabei wird der Quellcode auf diverse Aspekte überprüft. Bei nicht bestehender Überprüfung wird eine Meldung ausgegeben.\\
Diese Methode hat bisher die Fehlerquate um einiges verringert. Es ist jedoch immer noch möglich einen Fehler in die Produktion einzuspielen.  
Folgende Fehlerfälle können trotz pre-commit hooks eintreffen. 
\begin{itemize}
    \item Man commited direkt über Github.
    \item Das requirements.txt für pip ist nicht vollständig.
\end{itemize}
Derzeitig wird der Quellcode nach folgenden Zeichenkette überprüft:
\begin{itemize}
    \item ''import pdb'' - *.py: 
        pdb ist der Python Debugger. Aus Sicherheitsgründen darf dieser nicht in der Produktion verwendet werden.
    \item ''import ipdb'' - *.py:
        Gleiche Begründung wie ''import pdb''.
    \item ''print'' - *.py:
        Konsolenausgabe sind in der Produktion unerwünscht.
    \item ''console.log'' - *.js:
        Konsolenausgabe sind in der Produktion unerwünscht.
    \item ''debugger'' - *.js:
        Das Keyword debugger ist mir im Javascript nicht bekannt.
    \item 
\end{itemize}
Derzeitig wird der Quellcode mit folgenden tools überprüft:
\begin{itemize}
    \item jshint - *.js:
        jshint ist ein Code Qualitätstool, dass den Quellcode nach Fehlern und unschönen Programmiertechniken durchsucht.
    \item pyflakes - *.py:
        pyflakes überprüft den Quellcode nach potenziellen Fehlern.
    \item pep8 - *.py:
        pep8 ist ein Tool welches den Quellcode auf das Einhalten des pep8 Standards überprüft.
    \item sass - *.scss:
       sass ist eine Erweiterung von CSS3. Sass wir von uns nicht mehr verwendet. Diese üperprüfung ist somit nicht nötig. 
\end{itemize}

\subsection{Fehlererkennung}
Jede Djangoapplikation sendet im Fehlerfall automatisch ein Email an eine vorgebeben Email-Adresse mit einem ausführlichen Bericht. Darunter zählen jedoch nur Pythonfehler. Diese Funktionalität wird vom django Framework bereitgestellt. Wenn ein Fehler im settings.py auftritt, wird keine Fehlermeldung gesendet. Der Fehler bleibt unentdeckt bis jemand die Htmlseite aufruft. \\
Generell wird ein Fehler erst entdeckt wenn dieser ausgelöst wird. Das heisst, wenn ein Fehler nicht auf der lokalen Testumgebung ausgelöst und behoben werden kann, so wird mit zimmlicher Wahrscheinlichkeit der Besucher der Webapplikation den Fehler auslösen. \\
Fehler, die durch Serverprobleme verursacht werden, werden nicht speziel abgefangen oder kontrolliert, da diese relativ wenig vorkommen. 

\section{Gewünschter Soll-Zustand}
Die Probleme, die mit der jetztigen Qualitätssicherung auftreten sollen so weit wie möglich verhindert werden. Die meist auftretenden Fehler sollen weit weg vom Kunden entdeckt werden. Das heisst die Überprüfung müsste automatisiert und ständig auf einem externen System laufen.
\clearpage

        \chapter{Terminplanung}
            \begin{landscape}
            \begin{center}
\includegraphics[height=1\textwidth,angle=0]{./grafiken/terminplanung.pdf}
\end{center}

            \end{landscape}
        \chapter{Funktionsumfang}
            %!TEX root = ../dokumentation.tex
\section{Funktionsumfang}
Die Aufgabenstellung ist im Kapitel 1 detailiert beschrieben. \\
Es wurde eine Liste mit Funktionen erstellt, die für das erreichen des Projektziel notwendig sind.
\begin{itemize}
    \item Das Script soll täglich mittels Cronjob ausgelöst werden.
    \item Klonen eines Projekt aus GitHub. 
    \item Erstellen einer virtuellen Umgebung mittels virtualenvwrapper.
    \item Installieren des Projekts und deren Abhängigkeiten.
    \item Aufsetzen einer mysql-Datenbank. 
    \item Synchronisieren der Datenbank mit den Djangomodellen.
    \item Ausführen der Datenbankmigrationen.
    \item Überprüfung des Quellcodes anhand der vorgegebenen Validierungskriterien (siehe Abschnitt Validierungskriterien).
    \item Starten der Djangoapplikation
    \item Alle Berichte werden in eine Logdatei geschrieben. 
    \item Senden eines Berichts als Email an eine vordefinierte Adresse.
    \item Entfernen der Datenbank aus der lokalen Umgebung.
    \item Entfernen des überprüften Projekt und der virtuellen Umgebung aus der lokalen Umgebung.
\end{itemize}


\section{Validierungskriterien}
Folgende Kriterien müssen erfüllt sein, damit ein Djangoprojekt erfolgreich validiert wird:
\begin{itemize}
    \item Codeüberprüfung durch pep8 - *.py
    \item Codeüberprüfung durch pyflakes - *.py
    \item Codeüberprüfung durch jshint - *.
    \item Der Code darf die folgenden Zeichenfolgen nicht enthalten: ''console.log'', ''debugger'' - *.js
    \item Der Code darf die folgenden Zeichenfolgen nicht enthalten: ''print'', ''import pdb'', ''import ipdb'' - *.py
\end{itemize}

\section{Kriterien für die Erweiterbarkeit}
Damit das Script auch bequem erweitert werden kann, muss auf ein paar Dinge besonds geachted werden. \\
Darunter gehören eine modulare Aufteilung der einzelnen Prüfungsprozesse und eine Kapselung des Basiscodes. \\
Mit Einhaltung folgender Kriterien sollte das möglich sein:
\begin{itemize}
    \item Die Prüfungsscripts sind in einem separatem Order gespeichert.
    \item Der Bezug auf die Prüfungsscripts wird nur in einer Datei gehalten. 
    \item Das Layout für den Bericht, der über Email versendet wird, wird in einer separaten Datei gehalten.
    \item Der Prüfungsbericht soll so gestaltet sein, dass die Resultate auch bei sehr vielen Projekten oder sehr vielen Prüfungsprozessen noch überschauber ist.
\end{itemize}

\clearpage
\section{Prüfungsbericht}
Der Prüfungsbericht wird nach Abschluss der Überprüfung aller Projekte als Email versendet. Im Bericht sollen alle Diagnosen der Projekte aufgezählt sein.
Der Bericht wird in eine Gesamtsübersicht und in einer Detailübersicht aufgeteilt. Die Gesamtübersicht wird immer angezeigt und stellt alle Projekte und eine grobe Erfolgsanzeige in einer Tabelle dar. \\
Die Detailanzeige wird nur dan angezeigt, wenn bei der Überprüfung eines Projekts Mängel auftauchen. Jedes Projekt, dass die Überprüfung nicht erfolgreich abgeschlossen hat, bekommt seine eingen Detailansicht.
In der Detailansicht wird jediglich darauf hingewiesen, welcher Prüfungsprozess einen Fehler zurückgibt. Die originale Fehlermeldung wird nicht angezeigt. 

\section{Modulstruktur des Scripts}
Wie bereits in dem Abschnit ''Kriterien für die Erweiterbarkeit'' erwähnt, soll das Script mit Prüfungsprozessen erweiterbar sein ohne den Basiscode zu verändern. Im folgenden Abschnitt wird genauer auf die Aufteilung der einzelnen Module eingegangen. \\

\section{Scriptablauf}
\subsection{Abarbeitung der Djangoprojekt}
Es werden alles Djangoprojekt nacheinander überprüft. Am schluss wird der Bericht per Mail versendet.
Im nachfolgendem Flussdiagramm ist ersichtlich, wie die Projekte nacheinander abgearbeited werden: \\

\subsection{Arbeitsablauf der Testcases in einem Djangoprojekt}
Die Testcases werden nacheinander geladen und aufgerufen. Nach jedem abgeschlossenen Testcase wird das entsprechnde Resultat für die Auswertung zwischengespeichert. 
Die Überprüfung eines Djangoprojekts ist abgeschlossen, wenn alle Testfälle durchgelaufen sind oder ein Testfall gescheitert ist, welcher als erfolgsabhängig eingestuft wurde. 
Im nachfolgendem Flussdiagramm ist ersichtlich, wie die Testcases in einem Projekt nacheinander abgearbeited werden: \\ 


        \chapter{Implementierung}
            \section{Zwischentests}
Während das Script Implementiert wird, wird das ''allink-project'' als Testproject verwendet. Das ''allink-project'' ist das Standardtemplate für unsere Djangoprojekte. Die akutelle Version des Projekts ist unter dem Commit: 

\section{Funktionalität des Scripts}
\subsection{Testcases auswählen}
Man kann für jedes Projekt entweder eine Liste der Testfälle erstellen, die ausgeführt werden sollen oder eine Liste der Testfälle, die nicht ausgeführt werden sollen.
So kann man für jedes Projekt die Testfälle individuell zusammenstellen.

\section{Fehlerbehandlung}
Jede Funktion im Script wird auf Fehler überprüft. Das Script sollte unter keinen Umständen zum absturz kommen. 
Im Fehlerfall werden die restlichen Testfälle übersprungen und der Bericht wird noch erstellt und versendet. \\
Jeder Testfall hat 2 Fehler, die geworfen werden können: 
\begin{itemize}
    \item CriticalError - kritischer Fehler, führt dazu dass die restlichen Testfälle einer Djangoapplikation ausgelassen werden, wird im Report rot dargestellt 
    \item NonCriticalError - nicht kritischeer Fehler, der Fehler wird zwar vermekrt, fürt aber nicht zum Abbruch, die restlichen Testfälle werden weiterhin ausgeführt, wird im Report orange dargestellt
\end{itemize}

\section{E-Mail versenden}
Das versenden von Emails ist mit der in Python integrierten Funktionalität gelöst worden. \\
Quelle der Informationen: \\ 
http://docs.python.org/library/smtplib.html \\
http://docs.python.org/library/email-examples.html \\

\section{Logging}
Ausser des Report, welcher per E-Mail versendet wird, werden auch Fehlermeldungen in eine Logdatei geschrieben. Die Logdateien befinden sich im Ordner ''logs''. Jede Projekt hat seine eigene Logdatei. Die Logs werden an den Dateien angehängt. Die Logs werden nich gelöscht. Man muss sie also bei bedarf manuell löschen. \\
Das Loggen wurde mit einer integrierten Funktionalität von Python gelöst. \\
Quelle der Informationen: \\
http://docs.python.org/library/logging.html
http://docs.python.org/howto/logging.html\#logging-basic-tutorial \\

\section{Subprozesse ausführen}
Externe Anwendungen werden als Subprozess im Script ausgeführt. Der Rückgabewert des Subprozesses wird überprüft und dementsprechend eine eigens definierter Fehlertext im Report ausgegeben.
Quelle der Informationen: \\
http://docs.python.org/library/subprocess.html


        \chapter{Testen}
            
\section{Testumgebung}
Das System, auf dem das Script getestet wird ist ein Mac OS X 10.7.3
Die Tests werden mit Python 2.7 ausgeführt.

\section{Testfälle für den Test}
Für den Test werden ein paar Dummytestfälle erstellt, die folgende Fehler werfen sollen:
\begin{itemize}
    \item CriticalError
    \item NonCriticalError 
    \item kein Fehler 
\end{itemize}

\section{Testen der Testfälle}
Da die Testfälle voneinander stark varieren, werden nur die im Abschnitt ''Testfälle für den Test'' erwähnten Testfälle in den Test einbezogen. Die restlichen Testfälls sind teilweise von externen tools oder von einer Internetverbindung abhängig. Das Resultat dieser Testfälle ist stark abhängig von dem Resultat der jeweiligen externen Applikaiton, welche im Testfall verwendet wird. Deshalb macht es mehr Sinn diese Testfälle nicht im Test auszulassen, weil man auf diese Weise mehrheitlich die externe Applikaiton testetn würde anstatt des Scripts.  

\section{Testablauf}
Beim Testen wird vor allem darauf geachtet, dass das Hauptscript funktioniert. Die einzelnen Testfälle sind sommit zweitrangig, da diese zu stark von einander varrieren. Ohne Testfälle kan aber das Hauptscript nicht überprüft werden. Desswegen werden die wichtigsten Testfälle mit in den Test genommen.
Es werden Testscenarien ausgesucht um die wichtigsten Features des Scripts zu überprüfen.
Der Test soll automatisiert ablaufen. Nach dem manuellen ausführen der Testapplikation, werden alle Testfälle nacheinander abgearbeitet. In der Konsolenausgabe wird dan mitgeteilt ob der Test erflogreich abgearbeitet werden konnte. Bei nichtbestehen des Tests wird auf den gescheiterten Testfall hingewiesen.  

\subsection{Testscenarien}
Es wurden realitätsbezogenen Scenarien ausgesucht. 
Folgende Testscenarien werden im Test behandelt:
\begin{enumerate}
    \item alle Testfälle sind erfolgreich 
    \item ein Testfall wirft einen kritischen Fehler zurück
    \item ein Testfall wirft einen nicht kritischen Fehler zurück
    \item ein Testfall wird ingonriert (exclude)
    \item es werden alles Testfälle zugelassen ausser einer (allowed\_test\_cases) 
\end{enumerate}

\section{Testkriterien}
Beim Testen werden folgende Kriterien getestet:
\begin{itemize}
    \item eine Djangoapplikation kann aus einem git Repository geklont werden
    \item es kann eine virtuelle Umgebung wird für die Djangoapplikation erstellt werden
    \item die Djangoapplikation kann sammt virtueller Umgebung wieder entfernt werden
\end{itemize}

\section{Testen des Reports}
Für den Report wird ein Testfall zusammengestellt, der kritische, wie auch nichtkritische Fehler darstellt, ingorierte Testfälle und ausgelassene Testfälle (ausgelassen wegen eines kritischen Fehlers) werden grau dargestellt. 
 Der Report muss auf mail.google.com angeschaut werden. Den es ist nicht vergleichbar wenn man eine HTML-Seite anschaut oder ein HTML-Mail in einer Mailapplikation betrachtet. HTML-Mails sind was Styling angeht, eingeschränkt. 

\clearpage
\subsection{Testvorgehen }
In diesem Scenario wird jeder Testfall erfolgreich abgeschlossen.
Handlung: \\
test\_report\_success.py ausführen \\
Reaktion: \\
Script wird ausgeführt und automatisch beendet, es wurde eine mail an kuzma@allink.ch versendet.\
Das E-Mail wird mit einem im Excel erstellten Report verglichen. Der Excelreport stellt das erwartete Ergebnis dar. 


\subsection{Testen der Testfälle}
Der Erfolg der Testfälle hängt stark von äusseren Faktoren ab. Mit den Unittest und dem Test des Report kann man zumindest ausgehen das Die Basislogik des Scripts funktioniert. 
Befor aber ein Testfall in die Produktion eingespielt wird, werden noch ein paar Projekte mit dem Testscript überprüft. Dafür wählt man Projekte, die man kennt oder sogar daran gearbeitet hat. 
Für den folgenden Test werden 3 von mir bekannte Projekte mit dem Script überprüft:
\begin{itemize}
    \item test-projekt: Das standard Template für unsere Djangoprojekte. Commit zum Zeitpunkt des Tests: 5116d7ea67746dbd109954206be839be8af4d833 
    \item zumsteg: Commit zum Zeitpunkt des Tests: aa873796518a14761599b364c147436762cc1fa1  
    \item looksswiss: Commit zum Zeitpunkt des Tests: 79be790bcdf2b289a6a6f51ec33259d587373010 
\end{itemize}
 
Anbei sind die Reports der 3 Projekte abgebildet:


    
    
    % \chapter{Arbeitsjournal}
    %     \input{../arbeitsjournal/arbeitsjournal.tex}
    
\end{document}
