%!TEX root = ../dokumentation.tex
\section{Analyse des Ist-Zustands}
Momentan werden unsere Djangoprojekte beim commiten (versionieren mit hilfen von git) mit sognennten ''pre-commit hooks'' analysiert. 
Dabei wird der Quellcode auf diverse Aspekte überprüft. Bei nicht bestehender Überprüfung wird eine Meldung ausgegeben.\\
Diese Methode hat bisher die Fehlerquate um einiges verringert. Es ist jedoch immer noch möglich einen Fehler in die Produktion einzuspielen.  
Folgende Fehlerfälle können trotz pre-commit hooks eintreffen. 
\begin{itemize}
    \item Man commited direkt über Github.
    \item 
\end{itemize}
Derzeitig wird der Quellcode nach folgenden Zeichenkette überprüft:
\begin{itemize}
    \item ''import pdb'' - *.py: 
        pdb ist der Python Debugger. Aus Sicherheitsgründen darf dieser nicht in der Produktion verwendet werden.
    \item ''import ipdb'' - *.py:
        Gleiche Begründung wie ''import pdb''.
    \item ''print'' - *.py:
        Konsolenausgabe sind in der Produktion unerwünscht.
    \item ''console.log'' - *.js:
        Konsolenausgabe sind in der Produktion unerwünscht.
\end{itemize}




\section{Präzisierung der Aufgabenstellung}


\subsection{Funktionsumfang}
\begin{itemize}
    \item Alle Berichte werden in eine Logdatei geschrieben. 
    \item Alle Berichte werden per Mail ein eine vorgegebene Email-Adresse versendet. 
\end{itemize}

\section{Definition Ziele}
In einer Sitzung mit der Projektleitung wurde der IST-Zustand besprochen und die Ziele definiert.
\clearpage
